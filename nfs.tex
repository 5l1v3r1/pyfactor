% Section on the number field sieve 
% gets included in the report.tex file 


Let $N$ be a positive composite integer that is not a prime power. The general number field sieve is a result of 30 years of evolution of same general idea: finding integers $x,y$ such that $x \not\equiv y$ mod $N$ but $x^2 \equiv y^ 2$ mod $N$. Once these integers are found, we know $N$ divides $(x - y)(x+y)$. Since $N$ is not prime by assumption, it is likely (about half the time) that one of $\text{gcd}(x-y,N)$ or  $\text{gcd}(x+y,N)$ is non-trivial. The first algorithm to leverage this idea was designed by John Dixon at Carleton University [ref dixon]. The key insight was that producing congruences of squares modulo $N$ could be done using a factor base of primes. The technique seeks integers with squares mod N are \textit{smooth} over those primes. This simple idea has been extrapolated into what is now known to the fastest factoring algorithm ever implemented, the number field sieve. 


%% quick description of dixons algorithm
\subsection{Dixon's Method}
  Let $F = \lbrace 2,3,...,p_t \rbrace$ be a set of consecutive prime numbers. An integer $z$ is “$F$-smooth” if $z = 2^{e_1} \cdots p_t^{e_t} $ where $e_1,...,e_t \in \mathbb{Z}^+$. Dixons big idea was to collect a set of integers $A = \lbrace x \in \Z \mid x^2 \text{ is }F\text{-smooth mod } N \rbrace $ with $s = \mid A \mid $. Then write these squares in the form $x_i^2 = 2^{e_{1i}}  \cdots p_{t}^{e_{ti}} \text{ mod } N$ for $i = 1,...,s$ and place them in a matrix as 

  $$
  \begin{pmatrix}
    e_{11} \text{ mod } 2 & e_{21} \text{ mod } 2 & \cdots & e_{t1} \text{ mod } 2 \\
    e_{12} \text{ mod } 2 & e_{22} \text{ mod } 2 & \cdots & e_{t2} \text{ mod } 2 \\
    \vdots  & \vdots  & \ddots & \vdots  \\
    e_{1s} \text{ mod } 2 & e_{2s} \text{ mod } 2 & \cdots & e_{ts} \text{ mod } 2 
   \end{pmatrix}
  $$

  If $s>t$ then there are more rows than colums. Then there exists a subset $U \subset A$ whos corresponding rows are linearly dependent, which means that they sum to the zero vector. This means that 

  $$
  (\prod_{x_i \in U}x_i)^2 \equiv \prod_{x_i \in U}x_i^2 \equiv 2^{2f_1} \cdots p_t^{2f_t} \equiv (2^{f_1} \cdots p_t^{f_t})^2 \text{ mod } N \text{ where } f_j = \frac{\sum_{x_i \in U} e_{ji}}{2} 
  $$

  Let $x= \prod_{x_i \in U}x_i $ and $y = 2^{f_1} \cdots p_t^{f_t}$, then $x^2 \equiv y^2$ mod $N$. 

%% Quick description of the quadratic sieve 
\subsection{Quadratic Sieve}
  The bottleneck of Dixon's Mathod is constructing the set $A = \lbrace x \in \Z \mid x^2 \text{ is }F\text{-smooth mod } N \rbrace$. Initially, this was done by randomly choosing integers less than $N$ and squaring them. This was radically improved in 1981 by Carl Pomerance with his idea to instead use the factor base to find smooth integers mod $N$. His idea was the combination of two key observations: When trying to find smooth squares mod N, if $x$ is an integer in the range $\sqrt{N} < x < \sqrt{N} + N^\delta $ for some small positive delta, then $x^2 - N \approx N^{1/2 + \delta}$. This means that if we let $f(x) = x^2 - N $, then as $x$ takes values starting from $\sqrt{N}$, then the values of $f(x)$ are small squares mod $N$. Secondly, if $x_1,x_2$ are a roots of $f(x)= x^2 - N$, then so are $x_1 + kp,x_2 + lp$ for every integer $k,l$. Keeping these two facts in mind the idea of the quadratic sieve is to initialize two arrays

  \begin{align*}
    R &= [\ceil*{{\sqrt{N}}},\ceil*{{\sqrt{N}}}+ 1, ..., \ceil*{\sqrt{N}} + N^\delta] \\
    \text{LogR} &= [\log(f(\ceil*{\sqrt{N}}) \text{ mod } N), \log(f(\ceil*{\sqrt{N}}+1) \text{ mod } N),..., \log(f(\ceil*{\sqrt{N}}  + N^\delta) \text{ mod } N)] 
  \end{align*}

  where $\text{LogR}[i] = \log(f(R[i]))$ for all $i$. Then for each $p_j$ in the factor base $F = \lbrace 2,3,...,p_t \rbrace $, iterate through $R$ checking if $p$ divides $f(R[i])$ mod $N$, if it does subtract $\log(p_j)$ from $\text{LogR}[i + kp_j]$ for all $0 \leq k < \floor*{\frac{N}{p_j}}$ and skip to the next prime $p_{j+1}$. Once this is done for all primes in the factor base, scan through the array LogR and look for values which are approximately $0$. These values are $F$-smooth mod $N$. This is because if a value satisfies $f(R[i]) = 2^{e_1} \cdots p_t^{e_t}$ then 
  $$ 
    \log(f(R[i])) - e_1\log(2) - ... - e_t \log(p_t) = \log(\frac{f(R[i])}{2^{e_1} \cdots p_t^{e_t}}) = \log(1) = 0
  $$

  Once these $F$-smooth integers $f(R[i])$ are found, they are placed in a matrix the same way as in Dixon's algorithm. 
   
%% Quick description of the quadratic sieve
\subsection{The Number Field Sieve}
  Followign the success of the Quadratic Sieve, researchers experiemented with various elements of the algorithm. One insight was that the polynomial $f(x) = x^2 - N $ used in the quadratic sieve doesn't necessarily have to be quadratic! The other key insight was that there are \hyperref[ring]{\textit{rings}} which have the similar notions of smoothness and may have "more" smooth elements that the regular integers. The following section describes such a ring. If the reader is confortable loosing some of the intuition behind the algorithm, they may choose to skip to section \ref{algo}
  \subsubsection{Algebraic Intuition}
    Let $f(x)$ be a monic irreducible polynomial of degree $d$ in $\mathbb{Z}[x]$. By the \hyperref[fun]{\textit{fundamental theorem of algebra}}, $f(x)$ has exactly $d$ complex roots $\theta_1, ... , \theta_d$. Choose one of these roots $\theta$, then one may consider the set $\Z[\theta]$ of all $\Z$-linear combinations of the elements $\lbrace 1,\theta,\theta^2,...,\theta^{d-1} \rbrace $. Given $g =  a_0 + a_1\theta + ... + a_n\theta^n $ and $ h= b_0 + b_1 \theta + ... + b_m \theta^m $ in $ \Z[\theta]$ with $m < n < d$, let $G(x),H(x) \in \mathbb{Z}[x] $ such that $G(\theta) = g, H(\theta) = h $. Note that we may write $G(x)H(x) = C(x)f(x) + D(x)$ where $C(x), D(x) \in Z[x]$ and $\deg(f(x)) >  \deg (D(x))$. Therefore $ G(\theta) H(\theta) = C(\theta)f(\theta) + D(\theta) = 0 + D(\theta) = D(\theta) $. If we let $d = D(\theta)$, then with the operations  
    \begin{align*}
    g+ h &=  (a_0 + b_0) + (a_1 + b_1) \theta + \dots + (a_n + b_n )\theta^n \\
    g \cdot  h &= d
    \end{align*}
    the set $\Z[\theta]$ forms a ring. This ring is actually a subring of the \hyperref[field]{\textit{field}} $\Q(\theta)$ which is isomorphic to $\Q[x]/(f)$ and thus the two operations decribed above are inherited from the field $\Q(\theta)$. Besides making the following definition of a "norm" function on $\Q(\theta)$, this fact will generally be ignored for the purposes of this report. Recall that $f$ has precisely $d$ roots $\theta_1, ... , \theta_d$ over $\mathbb{C}$ and we chose one of them and called it $\theta$. For $i = 1,...,d$ let $\sigma_i : \Q(\theta) \longrightarrow \Q(\theta_i)$ be defined by the action; $\sigma_i(\Q) = \Q$ and $\sigma_i(\theta) = \theta_i$. It turns out that these $\sigma_i$ are precisely all the embedding of $\Q(\theta) $ into $\mathbb{C}$, but again this can be ignored. Define the "norm" function $N$ on $\Q(\theta)$ to be a set map $\Q(\theta) \overset{N}{\longrightarrow} \mathbb{C}$ with action $\alpha \mapsto \sigma_1(\alpha) \sigma_2(\alpha), ... , \sigma(\alpha)$. It is a standard result from algebra that $N$ is a multiplicative function which maps $Q(\theta)$ to $\Z$ and thus likewise for $\Z[\theta]$. This will norm function $N$ will be a very important link between the two rings $\Z$ and $\Z[\theta]$. \\

    Since $\Z[\theta]$ is a ring, we may consider \hyperref[ideal]{\textit{ideals}} in $Z[\theta]$. Furthermore, since $\Z[\theta]$ is in fact a \hyperref[dedekind]{\textit{Dedekind domain}}, an ideal $I$ of $\Z[\theta]$ may be uniquely factored, up to order, as the product $I = \mathfrak{p}_1^{e_1} \mathfrak{p}_2^{e_2} \cdots \mathfrak{p}_t^{e_t}$ of \hyperref[primeideal]{\textit{prime ideals}} $ \mathfrak{p}_1, \mathfrak{p}_2, ... , \mathfrak{p}_t \in \Z[\theta]$. Two more facts are necissary before it will become apparent why all this algebra is required. We may also define the norm $N$ on an ideal of $I \in \mathbb{Z}[\theta]$ as $N(I) = [\mathbb{Z}[\theta]: I]$. That is, the number of cosets of $I$ in $\mathbb{Z}[\theta]$. It turns out that this definition agrees with our earlier definition in the sense that $\mid N(\alpha)\mid  = N(\langle \alpha \rangle )$. Secondly, if $\mathfrak{p}$ is an ideal of $\mathbb{Z}[\theta]$ such that $N(\mathfrak{p}) = p$ for some prime $p$, then $\mathfrak{p}$ is a prime ideal. Conversly, if $\mathfrak{p}$ is a prime ideal of $\mathbb{Z}[\theta]$, then $N(\mathfrak{p}) = p^e$ for prime $p \in \Z$ and possitve exponent $e$. The point is of all this is, for $\alpha \in \mathbb{Z}[\theta]$ with factorization $\alpha = \mathfrak{p}_1^{e_1} \mathfrak{p}_2^{e_2} \cdots \mathfrak{p}_t^{e_t}$, then

    \begin{align*}
      \norm{N(\alpha)} & = N(\langle \alpha \rangle) \\
      &=N(\mathfrak{p}_1^{e_1} \mathfrak{p}_2^{e_2} \cdots \mathfrak{p}_t^{e_t}) \\
      &=N(\mathfrak{p}_1)^{e_1}N(\mathfrak{p}_2)^{e_2} \cdots N(\mathfrak{p}_2)^{e_t} \\
      &=(p_1^{f_1})^{e_1}(p_2^{f_2})^{e_2} \cdots (p_t^{f_t})^{e_t} \\
      &=p_1^{f_1 + e_1}p_2^{f_2 + e_2} \cdots p_t^{f_t + e_t}
    \end{align*}
    for some (not necissarly destinct) primes $p_1, p_2, ..., p_t$ and possitive exponents $f_1,f_2,...,f_t$. Essentially, we have been able to relate questions about factorization in $\Z[\theta]$ to questions of factorization in $\Z$. The high level idea is to choose a set $\A$ of prime ideals in $\Z[\theta]$, called an "algebraic factor base" which we will use analogously to our factor base in the integers. The idea is to try an find pairs $(a,b)$ for which the element $a + b \theta$ has a principal ideal $\langle a+ b \theta \rangle$ that factors as the product of prime ideals which are in $\A$. We call this being smooth over $\A$. \\

    Two technical problems remains - storing representations of prime ideals in $\Z[\theta]$ on a computer and determining whether an algebraic element of $\Z[\theta]$ is a square. \\ 

    The former can be overcome by only comprising one's factor base of prime ideals $\mathfrak{p}$ which satisfy $N(\mathfrak{p})=p$ for some prime $p$. One can show that these ideals, called "first degree prime ideals", are the only prime ideals apearing in the prime ideal factorization of a principal idea $\langle a + b \theta \rangle $ for coprime integers $a,b\in \Z$ in $\Z[\theta]$. What makes these types of ideals perfect for computing is that they are in natural bijection with the set of all pairs $(r,p)$ where $p$ is a prime and $r$ satisfying $f(r) \equiv 0$ mod $p$. Furthermore, there is a easy to check condition for determining whether a first degree prime ideal occurs in the ideal factorization of $\langle a + b \theta \rangle $. Specifically, a first degree prime ideal with representation $(r,p)$ occurs in the ideal factorization of an element $\langle a + b \theta \rangle $ if and only if $a \equiv -br $ mod $p$. To summarize, finding an element $\langle a + b \theta \rangle $ that is smooth over an algebraic factor base of first degree prime ideals of $\Z[\theta]$ amounts to finding an element $a + b \theta $ such that the integer $N(a + b \theta)$ factors completely over the primes occurring in the $(r,p)$ pairs corresponding to the first degree prime ideals in the algebraic factor base. \\

    To determine whether an element $\beta$ is a square in $\Z[\theta]$, 


\subsubsection{Algorithm} \label{algo}

 
\begin{algorithm} 
  \caption{The general number field sieve to factorize an integer $N$}
  \begin{algorithmic}[1]
    \State \Return{$N$}
  \end{algorithmic} 
\end{algorithm} 

\subsubsection{Implementation}


\begin{minted}[mathescape,
               linenos,
               numbersep=5pt,
               gobble=2,
               frame=lines,
               framesep=2mm]{csharp}
  /*
  Here will be the python code.
  use xelatex -shell-escape report.tex to compile 
  */

\end{minted}



